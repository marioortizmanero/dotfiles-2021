\documentclass{article}

\usepackage[utf8]{inputenc}

% latex in spanish
\usepackage[spanish]{babel}

% code syntax highlighting
\usepackage{minted}
\usepackage{xcolor}
\usemintedstyle{vs}
\definecolor{bg}{HTML}{f8f8f8}
\setminted{
    frame=lines,
    framesep=2mm,
    bgcolor=bg,
    fontsize=\footnotesize,
    linenos
}

% plotting
\usepackage{graphicx}

% references
\usepackage{hyperref}
\usepackage{nameref}

% quotes con "abc"
\usepackage{csquotes}
\MakeOuterQuote{"}

\usepackage{geometry}
\geometry{a4paper, portrait, margin=1.2in}

\usepackage{pgfplots} % gráficos
\usetikzlibrary[patterns] % gráficos con patrones
\pgfplotsset{width=11cm, compat=1.8}

\graphicspath{ {./images/} }

\title{Título}
\author{Mario Ortiz Manero\\NIA \and Other\\NIA}
\date{Mes de Año}

% Comando para código inline
\newcommand{\code}[1] {\mintinline{text}{#1}}
\newcommand{\image}[1] {
  \begin{figure}[H]
    \centering
    \includegraphics[width=\textwidth]{#1}
  \end{figure}
}

\begin{document}

\maketitle
\newpage

\bigskip

% Índice
\setcounter{tocdepth}{2}
\tableofcontents
\newpage

% Márgenes entre párrafos sin \medskip
\setlength{\parskip}{0.75em}
% \setlength{\parindent}{0pt}

% Empieza en cero
% \setcounter{section}{-1}

% Saltos de línea siempre para evitar overflows.
% \pretolerance=10000

\section{Introducción} \label{intro}
Referencia a la bibliografía \cite{nombre}. Código \code{inline}. Bloque de
código:

\begin{minted}{rust}
fn main() {
    println!("Hello, World!");
}
\end{minted}

\href{https://google.com/}{Hipervínculo}, referencia con nombre a
\nameref{intro}, y con el número: sección \ref{intro}.

% Bibliografía
\newpage
\begin{thebibliography}{9} \label{bibliografia}
\bibitem{nombre} Nombre de Referencia
\end{thebibliography}

\end{document}
